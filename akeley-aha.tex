\documentclass[11pt]{article}
\usepackage[letterpaper, portrait, margin=4cm]{geometry}

\usepackage{enumitem}
\usepackage{amsmath}
\usepackage{placeins}
\usepackage{graphicx}
\usepackage{caption}
\usepackage[parfill]{parskip}
\newcommand{\solution}[1]{{{\color{blue}{\bf Solution:} {#1}}}}
\usepackage[usenames,dvipsnames,svgnames,table,hyperref]{xcolor}


\begin{document}
\title{Contributions to Aetherling}
\author{David Akeley}
\maketitle

\begin{abstract}
This is a list of my contributions to the Aetherling project. I start
with a section summarizing tasks I worked on, then I expand on the
tasks in further sections.
\end{abstract}

\section{Summary}

\begin{enumerate}

\item Line Buffer Specifications

I proposed a new specification of Aetherling's line buffer node and
wrote a document (``The Line Buffer Manifesto'') describing the
benefits of this redesign. The previous line buffer design was hard to
parallelize due to difficult-to-satisfy constraints on its parameters'
divisibility, and also did not support downsampling (``stride'').
The redesign addresses these issues.

\item Functional Simulator

I wrote a functional simulator for Aetherling, which includes a
simulation of the intended behavior of the redesigned line
buffer. This allows the user to test the functionality of an
Aetherling DAG using pure Haskell code.

\item Demonstration Apps

I designed two demonstration Aetherling DAGs: Gaussian blur (7 $\times$
7 stencil) and mipmap generation. These DAGs can be tested using the
functional simulator along with a library (\texttt{ImageIO.hs}) for
converting between simulator values and png images on disk. These apps
also demonstrate the applications of the redesigned line buffer.

\item Simplifying Ops

An Op is a node of an Aetherling DAG (along with its children,
sometimes). Previously there were multiple ways to express the same
functionality. I simplified the Ops to minimize redundancy.

\item Helper Functions

I wrote some Haskell helper functions for creating Ops and simple
patterns of Ops. These helpers check for invalid parameters and
substitute for functionality lost through the Op simplification.

\item 

\end{enumerate}

For example, elementwise addition of two
4-arrays-of-int could be expressed as an \texttt{Add} of 4-arrays or as
a \texttt{MapOp 4} over a scalar \texttt{Add}.

\end{document}

