\documentclass[11pt]{article}
\usepackage[letterpaper, portrait, margin=4cm]{geometry}

\usepackage{enumitem}
\usepackage{amsmath}
\usepackage{placeins}
\usepackage{graphicx}
\usepackage{caption}
\usepackage[parfill]{parskip}
\newcommand{\solution}[1]{{{\color{blue}{\bf Solution:} {#1}}}}
\usepackage[usenames,dvipsnames,svgnames,table,hyperref]{xcolor}


\begin{document}
\title{Contributions to Aetherling}
\author{David Akeley}
\maketitle

\begin{abstract}
This is a list of my contributions to the Aetherling project. I start
with a section summarizing tasks I worked on, then I expand on the
tasks in further sections.
\end{abstract}

\section{Summary}

\begin{enumerate}

\item Line Buffer Specifications

I proposed a new specification of Aetherling's line buffer node and
wrote a document (``The Line Buffer Manifesto'') describing the
benefits of this redesign. The previous line buffer design was hard to
parallelize due to difficult-to-satisfy constraints on its parameters'
divisibility, and also did not support downsampling (``stride'').
The redesign addresses these issues.

\item Functional Simulator

I wrote a functional simulator for Aetherling, which includes a
simulation of the intended behavior of the redesigned line
buffer. This allows the user to test the functionality of an
Aetherling DAG using pure Haskell code.

\item Demonstration Apps

I designed two demonstration Aetherling DAGs: Gaussian blur (7 $\times$
7 stencil) and mipmap generation. These DAGs can be tested using the
functional simulator along with a library (\texttt{ImageIO.hs}) for
converting between simulator values and png images on disk. These apps
also demonstrate the applications of the redesigned line buffer.

\item Simplifying Ops

An Op is a node of an Aetherling DAG (along with its children,
sometimes). Previously there were multiple ways to express the same
functionality. I simplified the Ops to minimize redundancy.

\item Helper Functions

I wrote some Haskell helper functions for creating Ops and simple
patterns of Ops. These helpers check for invalid parameters and
substitute for functionality lost through the Op simplification.

\item Ready-Valid Meta-Op

By default, Aetherling pipelines are composed of Ops representing
circuits with synchronous timing: they wait for a certain number of
warm-up cycles, then emit outputs on a repeating schedule. I designed
a `ReadyValid` op that represents the idea of wrapping a portion of an
Aetherling DAG with a ready-valid interface.  I modified the compose
operators (`|&|` and `|>>=|`) to properly handle the `ReadyValid` Op.
The user is prevented from composing an Op with ready-valid timing
with one with synchronous timing, and, when composing two ready-valid
ops in sequence, the throughput-matching behavior of the Aetherling
type system is suppressed.

\item ComposePar Retiming

Aetherling includes a `ComposePar` Op that represents placing circuits
(child ops) in parallel. The user is not required to ensure that each
parallel path has the same latency (sequential latency -- the count of
the number of register delays along a path). I wrote a pass that walks
an Aetherling DAG searching for `ComposePar` ops, modifying its child
ops if needed such that all paths have the same latency. The pass
finds an optimal solution that minimizes the number of register bits
added to the circuit.

\item Fractional Underutilization and Phase



\item Tests Written

I gained a lot of experience writing tests as part of my work
on the Aetherling project. These tests include:

\begin{enumerate}

\end{enumerate}

\end{enumerate}

For example, elementwise addition of two
4-arrays-of-int could be expressed as an \texttt{Add} of 4-arrays or as
a \texttt{MapOp 4} over a scalar \texttt{Add}.

\end{document}

